% Options for packages loaded elsewhere
\PassOptionsToPackage{unicode}{hyperref}
\PassOptionsToPackage{hyphens}{url}
%
\documentclass[
]{book}
\usepackage{lmodern}
\usepackage{amssymb,amsmath}
\usepackage{ifxetex,ifluatex}
\ifnum 0\ifxetex 1\fi\ifluatex 1\fi=0 % if pdftex
  \usepackage[T1]{fontenc}
  \usepackage[utf8]{inputenc}
  \usepackage{textcomp} % provide euro and other symbols
\else % if luatex or xetex
  \usepackage{unicode-math}
  \defaultfontfeatures{Scale=MatchLowercase}
  \defaultfontfeatures[\rmfamily]{Ligatures=TeX,Scale=1}
\fi
% Use upquote if available, for straight quotes in verbatim environments
\IfFileExists{upquote.sty}{\usepackage{upquote}}{}
\IfFileExists{microtype.sty}{% use microtype if available
  \usepackage[]{microtype}
  \UseMicrotypeSet[protrusion]{basicmath} % disable protrusion for tt fonts
}{}
\makeatletter
\@ifundefined{KOMAClassName}{% if non-KOMA class
  \IfFileExists{parskip.sty}{%
    \usepackage{parskip}
  }{% else
    \setlength{\parindent}{0pt}
    \setlength{\parskip}{6pt plus 2pt minus 1pt}}
}{% if KOMA class
  \KOMAoptions{parskip=half}}
\makeatother
\usepackage{xcolor}
\IfFileExists{xurl.sty}{\usepackage{xurl}}{} % add URL line breaks if available
\IfFileExists{bookmark.sty}{\usepackage{bookmark}}{\usepackage{hyperref}}
\hypersetup{
  pdftitle={Matemáticas para investigadores nivel uno},
  pdfauthor={John Jairo Estrada Álvarez},
  hidelinks,
  pdfcreator={LaTeX via pandoc}}
\urlstyle{same} % disable monospaced font for URLs
\usepackage{longtable,booktabs}
% Correct order of tables after \paragraph or \subparagraph
\usepackage{etoolbox}
\makeatletter
\patchcmd\longtable{\par}{\if@noskipsec\mbox{}\fi\par}{}{}
\makeatother
% Allow footnotes in longtable head/foot
\IfFileExists{footnotehyper.sty}{\usepackage{footnotehyper}}{\usepackage{footnote}}
\makesavenoteenv{longtable}
\usepackage{graphicx,grffile}
\makeatletter
\def\maxwidth{\ifdim\Gin@nat@width>\linewidth\linewidth\else\Gin@nat@width\fi}
\def\maxheight{\ifdim\Gin@nat@height>\textheight\textheight\else\Gin@nat@height\fi}
\makeatother
% Scale images if necessary, so that they will not overflow the page
% margins by default, and it is still possible to overwrite the defaults
% using explicit options in \includegraphics[width, height, ...]{}
\setkeys{Gin}{width=\maxwidth,height=\maxheight,keepaspectratio}
% Set default figure placement to htbp
\makeatletter
\def\fps@figure{htbp}
\makeatother
\setlength{\emergencystretch}{3em} % prevent overfull lines
\providecommand{\tightlist}{%
  \setlength{\itemsep}{0pt}\setlength{\parskip}{0pt}}
\setcounter{secnumdepth}{5}
\usepackage{booktabs}
\usepackage[]{natbib}
\bibliographystyle{plainnat}

\title{Matemáticas para investigadores nivel uno}
\author{John Jairo Estrada Álvarez}
\date{2022-01-18}

\begin{document}
\maketitle

{
\setcounter{tocdepth}{1}
\tableofcontents
}
\hypertarget{introducciuxf3n}{%
\chapter{Introducción}\label{introducciuxf3n}}

\hypertarget{ggb-elementSuperficieJJA2021}{}

\hypertarget{justificaciuxf3n}{%
\section{Justificación}\label{justificaciuxf3n}}

El hombre está en busca de cuantificar todo y la medicina no es un universo aparte, son incontables los trabajos realizados entre médicos, matemáticos, ingenieros, físicos haciendo que la ciencia progrese, con este curso y venideros si es posible, se quieren dar los primeros pasos para que las personas involucradas con las ciencias de la salud o de otras ramas se acerquen a la matemática y la apliquen en el momento oportuno, al adquirir los conceptos básicos se puede ir avanzando en temas un poco más avanzados e igualmente ver la utilidad en su quehacer diario, y a lo mejor adquirir las bases para realizar investigaciones en el área de la medicina, hay que tener en cuenta que esto lleva tiempo.

\hypertarget{objetivo-general}{%
\section{Objetivo General}\label{objetivo-general}}

Solución de problemas relacionados con las ciencias de la salud y de otros contextos(economía, etc), es decir, adquirir las nociones básicas de los fundamentos matemáticos y su relación con diferentes campos.

\hypertarget{obejtivo-especuxedficos}{%
\subsection{Obejtivo Específicos}\label{obejtivo-especuxedficos}}

\begin{itemize}
\item
  Aplicar las propiedades fundamentales de los operadores matemáticos.
\item
  Obtener destrezas básicas prácticas para la resolución de ecuaciones.
\item
  Incorporar destrezas básicas para analizar funciones.
\item
  Interpretar el concepto de derivada y razón de cambio y su uso práctico.
\item
  Asimilar fundamentos del enfoque matemático y su aplicación en otros campos como el de la salud, y su relación estrecha con la generación de conocimiento.
\item
  Reconocer comportamientos y/o modelos matemáticos inmersos en cualquier proceso en ciencias de la salud o en procesos administrativo.
\item
  Aplicar en forma eficaz las herramientas suministradas para la solución de situaciones problémicas.
\end{itemize}

\hypertarget{presentaciuxf3n-del-contenido}{%
\section{Presentación del contenido}\label{presentaciuxf3n-del-contenido}}

\begin{itemize}
\tightlist
\item
  \textbf{Capítulo 1}

  \begin{itemize}
  \tightlist
  \item
    Conceptos Introductorios
  \item
    Teoría de conjuntos.
  \item
    Conjuntos numéricos.

    \begin{itemize}
    \tightlist
    \item
      Tiempo de realización 2 horas 1 secciones
    \end{itemize}
  \end{itemize}
\item
  \textbf{Capítulo 2}

  \begin{itemize}
  \tightlist
  \item
    Operaciones y propiedades de los números reales.
  \item
    Inecuaciones.
  \item
    Potenciación y radicales.
  \item
    Ecuaciones lineales y cuadráticas.
  \item
    Ecuaciones polinomicas de grado mayor a dos.

    \begin{itemize}
    \tightlist
    \item
      Tiempo de realización 8 horas 4 secciones
    \end{itemize}
  \end{itemize}
\item
  \textbf{Capítulo 3}

  \begin{itemize}
  \tightlist
  \item
    Sistemas de ecuaciones lineales.
  \item
    Sistema de ecuaciones no lineales.
  \item
    Solución numérica de ecuaciones no lineales.

    \begin{itemize}
    \tightlist
    \item
      Tiempo de realización 8 horas 6 secciones
    \end{itemize}
  \end{itemize}
\item
  \textbf{Capítulo 4}

  \begin{itemize}
  \tightlist
  \item
    Planteo y solución de problemas
  \item
    Regla de tres simple directa e inversa.
  \item
    Regla de tres compuesta.

    \begin{itemize}
    \tightlist
    \item
      Tiempo de realización 8 horas 3 secciones
    \end{itemize}
  \end{itemize}
\item
  \textbf{Capítulo 5}

  \begin{itemize}
  \tightlist
  \item
    Concepto de función.
  \item
    Tipos de funciones.
  \item
    Operaciones entre funciones.
  \item
    Regresión lineal.

    \begin{itemize}
    \tightlist
    \item
      Tiempo de realización 6 horas 4 secciones
    \end{itemize}
  \end{itemize}
\item
  \textbf{Capítulo 6}

  \begin{itemize}
  \tightlist
  \item
    Talleres de evaluación.

    \begin{itemize}
    \tightlist
    \item
      Tiempo de realización 8 horas 4 secciones
    \end{itemize}
  \end{itemize}
\end{itemize}

\textbf{Total 40 horas}

\hypertarget{conceptos-introductorios}{%
\chapter{Conceptos introductorios}\label{conceptos-introductorios}}

\hypertarget{teoruxeda-de-conjuntos}{%
\section{Teoría de conjuntos}\label{teoruxeda-de-conjuntos}}

\hypertarget{conjuntos-numuxe9ricos}{%
\section{Conjuntos numéricos}\label{conjuntos-numuxe9ricos}}

\hypertarget{operaciones-y-propiedades-de-los-nuxfameros-reales}{%
\chapter{Operaciones y propiedades de los números reales}\label{operaciones-y-propiedades-de-los-nuxfameros-reales}}

\hypertarget{inecuaciones}{%
\section{Inecuaciones}\label{inecuaciones}}

\hypertarget{potenciaciuxf3n-y-radicales.}{%
\section{Potenciación y radicales.}\label{potenciaciuxf3n-y-radicales.}}

\hypertarget{ecuaciones-lineales-y-cuadruxe1ticas.}{%
\section{Ecuaciones lineales y cuadráticas.}\label{ecuaciones-lineales-y-cuadruxe1ticas.}}

\hypertarget{ecuaciones-polinomicas-de-grado-mayor-a-2.}{%
\section{Ecuaciones polinomicas de grado mayor a 2.}\label{ecuaciones-polinomicas-de-grado-mayor-a-2.}}

\hypertarget{sistemas-de-ecuaciones-lineales.}{%
\chapter{Sistemas de ecuaciones lineales.}\label{sistemas-de-ecuaciones-lineales.}}

\hypertarget{sistemas-de-ecuaciones-no-lineales.}{%
\section{Sistemas de ecuaciones no lineales.}\label{sistemas-de-ecuaciones-no-lineales.}}

\hypertarget{soluciuxf3n-numuxe9rica-de-ecuaciones-no-lianeales.}{%
\section{Solución numérica de ecuaciones no lianeales.}\label{soluciuxf3n-numuxe9rica-de-ecuaciones-no-lianeales.}}

\hypertarget{planteo-y-soluciuxf3n-de-problemas}{%
\chapter{Planteo y solución de problemas}\label{planteo-y-soluciuxf3n-de-problemas}}

\hypertarget{regla-de-tres-simple-e-inversa}{%
\section{Regla de tres simple e inversa}\label{regla-de-tres-simple-e-inversa}}

\hypertarget{regla-de-tres-compuesta}{%
\section{Regla de tres compuesta}\label{regla-de-tres-compuesta}}

\hypertarget{concepto-de-relaciuxf3n}{%
\chapter{Concepto de relación}\label{concepto-de-relaciuxf3n}}

\hypertarget{concepto-de-funciuxf3n}{%
\section{Concepto de función}\label{concepto-de-funciuxf3n}}

\hypertarget{tipos-de-funciones}{%
\section{Tipos de funciones}\label{tipos-de-funciones}}

\hypertarget{operaciones-entre-funciones}{%
\section{Operaciones entre funciones}\label{operaciones-entre-funciones}}

\hypertarget{regresiuxf3n-linneal}{%
\section{Regresión linneal}\label{regresiuxf3n-linneal}}

\hypertarget{talleres-evaluativos}{%
\chapter{Talleres evaluativos}\label{talleres-evaluativos}}

  \bibliography{book.bib,packages.bib}

\end{document}
